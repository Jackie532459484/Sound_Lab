\documentclass{article}
\usepackage{graphicx} %package to manage images
\graphicspath{ {images/} }
\usepackage[rightcaption]{sidecap}
\usepackage{wrapfig}
\begin{document}
\title{Sound Lab \\ Physics SL 12} % Title
\author{Jie Ji} % Author name

\maketitle % Insert the title, author and date
\large

%%%%%%%%%%%%%%%%%%%%%%%%%%%%%%%%%%%%%%%%%%%%%%%%%%%%%%%%%%%%%%%%%%%%%%%%%%%%%%%%%%%%%%%%%%%%%%%%%%%%%%%%%%%%%%%%%%%%%%%%%%%
\section{Introduction}
In this experiment, we are testing the different wavelength of the different pitchfork we use to create a cycle.
Also to test the reliabilities of the result from comparing calculations results and real-life experiment results. 
In order to try out the exact point of the wave length by changing the water level in the measuring cylinder.
%%%%%%%%%%%%%%%%%%%%%%%%%%%%%%%%%%%%%%%%%%%%%%%%%%%%%%%%%%%%%%%%%%%%%%%%%%%%%%%%%%%%%%%%%%%%%%%%%%%%%%%%%%%%%%%%%%%%%%%%%%%
\section{Apparatus}
\subsection{results from calculations}
\begin{table}[htbp]
\begin{center}
\footnotesize
\begin{tabular}{lllll}
\toprule
Items   & Usage of the Item                                                         \\
\midrule
Large Measuring cylinder & For measuring the water level and contain water\\
Water & Use to change the water level in the Measuring cylinder\\
Ruler & To Measure the wavelength from the top of the cylinder\\
Pitchfork & To create sound for experiment\\
\bottomrule
\end{tabular}
\end{center}
  \caption{Name and the use of Apparatus}
  \label{tab:font-sizes}
\end{table}
%%%%%%%%%%%%%%%%%%%%%%%%%%%%%%%%%%%%%%%%%%%%%%%%%%%%%%%%%%%%%%%%%%%%%%%%%%%%%%%%%%%%%%%%%%%%%%%%%%%%%%%%%%%%%%%%%%%%%%%%%%%
\section{Hypothesis}
The real-life experiment would be holding the similar results as the calculations, this experiment cannot be fully accurate so i had to say similar. The determine of the distance is by using human judgements to figure out the perfect sound wave the experiment can get for each pitchfork. So i believe that the experiment can get as close as the result on calculations. 
%%%%%%%%%%%%%%%%%%%%%%%%%%%%%%%%%%%%%%%%%%%%%%%%%%%%%%%%%%%%%%%%%%%%%%%%%%%%%%%%%%%%%%%%%%%%%%%%%%%%%%%%%%%%%%%%%%%%%%%%%%%
\newpage
%%%%%%%%%%%%%%%%%%%%%%%%%%%%%%%%%%%%%%%%%%%%%%%%%%%%%%%%%%%%%%%%%%%%%%%%%%%%%%%%%%%%%%%%%%%%%%%%%%%%%%%%%%%%%%%%%%%%%%%%%%%
\section{Steps of the experiment}
\begin{table}[htbp]
\begin{center}
\footnotesize
\begin{tabular}{lllll}
\toprule
Steps       & Instructions\\
\midrule
Step one & Pour water inside the measuring cylinder\\
Step two & use a pitchfork and hit it against a soft object to create sound\\
Step three & while the pitchfork is ringing, locate the pitchfork on top of the measuring cylinder\\
Step four & Increase the water level and repeat steps two and three\\
Step five & stop the experiment until finding the greatest sound after changing the water level\\
Step six & measure the distance from the top of the measuring cylinder to the water level to find wavelength. \\
* Important * & Repeat steps one to six for all different Frequency of pitchfork.
\bottomrule
\end{tabular}
\end{center}
  \caption{Instructions of the experiment}
  \label{tab:font-sizes}
\end{table}
%%%%%%%%%%%%%%%%%%%%%%%%%%%%%%%%%%%%%%%%%%%%%%%%%%%%%%%%%%%%%%%%%%%%%%%%%%%%%%%%%%%%%%%%%%%%%%%%%%%%%%%%%%%%%%%%%%%%%%%%%%%
\section{Formulae}
$$V= f \times \lambda$$
$$ \text{Velocity = frequency \times wavelength} $$
%%%%%%%%%%%%%%%%%%%%%%%%%%%%%%%%%%%%%%%%%%%%%%%%%%%%%%%%%%%%%%%%%%%%%%%%%%%%%%%%%%%%%%%%%%%%%%%%%%%%%%%%%%%%%%%%%%%%%%%%%%%
\section{Data}
\subsection{results from calculations}
\begin{table}[htbp]
\begin{center}
\footnotesize
\begin{tabular}{lllll}
\toprule
 Frequency of Pitchfork (hz)        & wave speed (m/s)         & wavelength   (m)  \\
\midrule
  512   & 84.992        & 0.166 \\
    384   & 85.248         & 0.222   \\
   256  & 85.914        & 0.333    \\
\bottomrule
\end{tabular}
\end{center}
  \caption{Results after calculations}
  \label{tab:font-sizes}
\end{table}

\subsection{results from experiment}
\begin{table}[htbp]
\begin{center}
\footnotesize
\begin{tabular}{lllll}
\toprule
 Frequency of Pitchfork (hz)        & wave speed (m/s)         & wavelength   (m)  \\
\midrule
  512   & 84.9        & 0.165 \\
  384   & 85.2        & 0.215 \\
  256   & 85.9        & 0.320 \\
\bottomrule
\end{tabular}
\end{center}
  \caption{Results after real testing experiment}
  \label{tab:font-sizes}
\end{table}
\newpage
%%%%%%%%%%%%%%%%%%%%%%%%%%%%%%%%%%%%%%%%%%%%%%%%%%%%%%%%%%%%%%%%%%%%%%%%%%%%%%%%%%%%%%%%%%%%%%%%%%%%%%%%%%%%%%%%%%%%%%%%%%%
\section{Analysis}
\subsection{Data analysis}
As we compare two tables on the result section above, we can clearly sees the results are different compare to the calculations. The data seems to be lesser then the calculations, the result we get is not very accurate and contain many errors. Also when we use this data, we found out that the pitchfork might have some mistake on listing the frequency. For example the result are more likely to be 500 hz instead of 512. Due to these data, we understand that the result might not be very reliable from the experiment and the calculations, we need an ideal situation to do this experiment with precise equipment to have an better results. 
%%%%%%%%%%%%%%%%%%%%%%%%%%%%%%%%%%%%%%%%%%%%%%%%%%%%%%%%%%%%%%%%%%%%%%%%%%%%%%%%%%%%%%%%%%%%%%%%%%%%%%%%%%%%%%%%%%%%%%%%%%%
\subsection{Possible experiment error and data collection}
- It is not accurate enough to use a ruler to check find the wavelength, it contains error by eyes and not precise enough for the operations. \\ \\
- The perfect sound that can produce from a wavelength is totally depends on people's hearing, people's judgment. It is an mistake for doing so, because people cannot get the most accurate point, instead we are purely guessing for the better sound.
%%%%%%%%%%%%%%%%%%%%%%%%%%%%%%%%%%%%%%%%%%%%%%%%%%%%%%%%%%%%%%%%%%%%%%%%%%%%%%%%%%%%%%%%%%%%%%%%%%%%%%%%%%%%%%%%%%%%%%%%%%%
\section{Conclusion}
The experiment clearly demonstrates that the real-life situation cannot be easily determined by the calculations, or theoretically speaking. Because the calculations are all very ideal circumstance, and the result is not accurate. However, this experiment provides us proof that we can find a full wave sound by using this type of experiment. 

%%%%%%%%%%%%%%%%%%%%%%%%%%%%%%%%%%%%%%%%%%%%%%%%%%%%%%%%%%%%%%%%%%%%%%%%%%%%%%%%%%%%%%%%%%%%%%%%%%%%%%%%%%%%%%%%%%%%%%%%%%%
\end{document}
%%%%%%%%%%%%%%%%%%%%%%%%%%%%%%%%%%%%%%%%%%%%%%%%%%%%%%%%%%%%%%%%%%%%%%%%%%%%%%%%%%%%%%%%%%%%%%%%%%%%%%%%%%%%%%%%%%%%%%%%%%%
